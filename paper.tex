\documentclass[twoside,10pt]{article}


\usepackage[letterpaper, hmargin=1in, vmargin={1in, 1.2in}]{geometry}
\usepackage{titling}
\usepackage{authblk}
\usepackage{hyperref}

%\usepackage[backend=biber, style=alphabetic, maxbibnames=99, doi=false, isbn=false, url=false]{biblatex}
%\addbibresource{biblio.bib}
\usepackage[utf8]{inputenc}


\usepackage{amsmath,amssymb,amsthm}
\usepackage{enumitem}
\usepackage{xspace}
\usepackage{mathtools}
\usepackage{multirow}
\usepackage{makecell}
\usepackage{bm}
\usepackage{url}
\usepackage{tikz}
\usetikzlibrary{shapes}
\usetikzlibrary{cd}
\usepackage[braket]{qcircuit}
\usepackage{calc}
\usepackage{algpseudocode}
\usepackage{caption}
\usepackage{subcaption}
\usepackage{tikz}

\usepackage[titletoc,title]{appendix}

\renewcommand\Affilfont{\small}

\DeclareMathOperator{\cutrank}{cutrank}
\DeclareMathOperator{\rw}{rw}
\DeclareMathOperator{\girth}{girth}

\renewcommand{\algorithmicrequire}{\textbf{Input:}}
\renewcommand{\algorithmicensure}{\textbf{Output:}}


\usepackage[linesnumbered,ruled,vlined]{algorithm2e}
\SetKwProg{Procedure}{Procedure}{}{}
\SetKwInOut{Input}{Input}
\SetKwInOut{Output}{Output}

\newtheorem{theorem}{Theorem}
\newtheorem{lemma}[theorem]{Lemma}
\newtheorem{corollary}[theorem]{Corollary}
\theoremstyle{definition}
\newtheorem{definition}[theorem]{Definition}
\newtheorem{example}[theorem]{Example}
\newtheorem{construction}{Construction}
\newtheorem{fact}[theorem]{Fact}
\newtheorem{conjecture}[theorem]{Conjecture}
\newtheorem{open}{Open Question}
\theoremstyle{remark}
\newtheorem{remark}[theorem]{Remark}


\newcommand{\CZ}{\ensuremath{\mathrm{CZ}}\xspace}
\newcommand{\NCZ}{\Gamma_\mathrm{CZ}}
\newcommand{\NCO}{\Gamma_\mathrm{EC}}
\newcommand{\defiff}{\overset{\text{def}}{\iff}}
\newcommand{\Mdel}{\setminus}
\newcommand{\Mcon}{\mathbin{/}}


\title{The minimum of girth and cogirth in binary matroids}
\author{James Davies}
\affil{Department of Pure Mathematics and Mathematical Statistic, University of Cambridge\\\texttt{jgd37@cam.ac.uk}}
\author{Ryuhei Mori}
\affil{Graduate School of Mathematics, Nagoya University\\\texttt{mori@math.nagoya-u.ac.jp}}


\begin{document}
\begin{titlingpage}
\maketitle
\thispagestyle{empty}

\begin{abstract}
For any minor-closed class of binary matroids, the minimum of girth and cogirth is $O(\sqrt{n})$ where $n$ is a number of elements in a matroid.
\end{abstract}
\end{titlingpage}
\pagenumbering{arabic}

\section{Introduction}
\section{Preliminaries: Graph, matroids and isotropic systems}
\subsection{Graph}
\begin{definition}[Local complementation]
\end{definition}

\begin{definition}[Pivot transformation]
%For a graph $G=(V,\,E)$, and an edge $\{u,v\}\in E$, \textit{pivot transformation} $G\wedge\{u,v\}$ is defined by
%\begin{enumerate}
%\item Toggle edges between $N_{G}(b)\setminus\{c\}$ and $N_{G}(c)\setminus\{b\}$.
%\item Exchange the labels of $b$ and $c$.
%\end{enumerate}
\end{definition}

\subsection{Binary matroids}
\begin{definition}[Matroid]
Let $E$ be a finite set, and $\mathcal{I}\subseteq 2^E$ be a family of subsets.
Then, $(E,\,\mathcal{I})$ is \textit{independence system} if
\begin{enumerate}
\item $\varnothing\in\mathcal{I}$.
\item $\forall T\subseteq S\in\mathcal{I}$, $S\in\mathcal{I}\implies T\in\mathcal{I}$.
\end{enumerate}
An independence system $(E,\,\mathcal{I})$ is \textit{matroid} if
\begin{enumerate}[resume]
\item $\forall S,\,T\in\mathcal{I}$, $|S|>|T|\implies\exists e\in S\setminus T$ such that $T\cup\{e\}\in\mathcal{I}$.
\end{enumerate}
\end{definition}

\begin{definition}[Base and circuit]
For a matroid $M$, a maximal independent set is called a \textit{base} of $M$.
A minimal dependent set is called a \textit{circuit} of $M$.
\end{definition}

\begin{definition}[Dual]
For a matroid $M=(E,\,\mathcal{I})$, the \textit{dual} of $M$ is a matroid $M^*=(E,\,\mathcal{I}^*)$ where
\begin{align*}
\mathcal{I}^*&\coloneqq\left\{S\subseteq E\mid \text{$E\setminus S$ includes a base}\right\}.
\end{align*}
\end{definition}

\begin{definition}[Matroid minors]
Let $M=(E,\,\mathcal{I})$ be a matroid.
A \textit{deletion} $M\setminus e$ for $e\in E$ and a \textit{contraction} $M\mathbin{/} e$ for $e\in E,\, \{e\}\in\mathcal{I}$ is defined as 
\begin{align*}
M\setminus e&\coloneqq \left(E\setminus\{e\},\, \left\{S\setminus\{e\}\mid S\in\mathcal{I}\right\}\right)\qquad\forall e\in E\\
M\mathbin{/} e&\coloneqq \left(E\setminus\{e\},\, \left\{S\setminus\{e\}\mid S\cup\{e\}\in\mathcal{I}\right\}\right)\qquad\forall e\in E,\, \{e\}\in\mathcal{I}.
\end{align*}
A matroid $N$ is a \textit{matroid minor} of $M$ if $N$ is obtained from $M$ by deletions and contractions.
\end{definition}

\begin{lemma}
\begin{align*}
(M\setminus e)^* &= M^*\mathbin{/}e\\
(M\mathbin{/} e)^* &= M^*\setminus e
\end{align*}
\end{lemma}

\begin{definition}[Fundamental graph of matroid]
For a matroid $M=(E,\,\mathcal{I})$ and a base $B$ of $M$,
a fundamental graph $G_{M,B}$ is a bipartite graph $(E,\, F)$ defined as
\begin{align*}
F&\coloneqq\left\{\{b,c\} \mid b\in B,\, c\in E\setminus B,\, B\bigtriangleup\{b,c\}\in\mathcal{I}\right\}.
\end{align*}
\end{definition}
Note that $G_{M,B}=G_{M^*,E\setminus B}$.

\begin{definition}[Binary matroid]
A matroid $(E,\,\mathcal{I})$ is a \textit{binary matroid} if there exists a binary matrix $B\in\mathbb{F}_2^{k\times E}$ for some $k\in\mathbb{N}$ such that
\begin{align*}
\mathcal{I} &\coloneqq\left\{S\subseteq E\mid \text{columns of $B$ corresponding to $S$ are linearly independent}\right\}.
\end{align*}
Here, $B$ is called a \textit{representation matrix}.
Without loss of generality, we can assume that rows of a representation matrix are linearly independent.
In this paper, we assume that rows of representation matrices are always linearly independent.
\end{definition}

Let $M$ be a binary matroid with a representation matrix $B\in\mathbb{F}_2^{k\times n}$.
Then, $C\in\mathbb{F}_2^{\ell\times n}$ is a representation matrix of the dual matroid $M^*$
if and only if $\ell = n-k$ and $BC^T = 0$ where ${}^T$ stands for the transposition of a matrix.

Let $M$ be a binary matroid with a representation matrix $B\in\mathbb{F}_2^{k\times n}$.
Then, for any invertible matrix $U\in\mathbb{F}_2^{k\times k}$, $UB$ is also a representation matrix for $M$.
With the multiplication of $U$ and column permutation, we can assume that the representation matrix has a form
\begin{align*}
B&=\begin{bmatrix}
I_k&\mid& A
\end{bmatrix}
\end{align*}
where $A\in\mathbb{F}_2^{k\times (n-k)}$.
In this case, we can take
\begin{align*}
C&=\begin{bmatrix}
A^T&\mid& I_{n-k}
\end{bmatrix}
\end{align*}
as a representatiom matrix for the dual matroid $M^*$.
Let $S\subseteq E$ be a subset corresponding to the first $k$ rows of the representation matrix $B=\begin{bmatrix}I_k&\mid&A\end{bmatrix}$.
In this case, a fundamental graph $G_{M, S}=G_{M^*, E\setminus S}$ has an adjacency matrix $A$. 

\if0
Let $M$ be a binary matroid with a representation matrix $B\in\mathbb{F}_2^{k\times n}$.
Then, for any invertible matrix $U\in\mathbb{F}_2^{k\times k}$, $UB$ is also a representation matrix for $M$.
Conversely, if $B$ and $B'$ represent the same binary matroid $M$, there exists an invertible matrix $U$ such that $B' = UB$.
Hence, the row space of representatiom matrices is uniquely determined from $M$.
This linear subspace is called a \textit{cut space} of $M$.

\begin{definition}[Cycle space and cycle matrix of a binary matroid]
For a binary matroid $M$, a linear space spanned by all circuits of $M$ is called a \textit{cycle space} of $M$.
A matrix $C\in\mathbb{F}_2^{\ell\tims k}$ whose row space is a cycle space of $M$ is called a \textit{cycle matrix}.
\end{definition}

Let $M$ be a binary matroid with a representation matrix $B$ and a cycle matrix $C$.
Then, the matrices $B$ and $C$ are cycle and representation matrices of the dual matroid $M^*$, respectively.
\fi

\begin{lemma}[Pivot transformation]
Let $M=(E,\,\mathcal{I})$ be a binary matroid and $B$ be a base of $M$.
Let $G_{M,B}$ be a fundamental graph.
Let $\{b,c\}$ be an edge of $G_{M,B}$ for $b\in B$ and $c\in E\setminus B$.
Then, $G_{M,B\bigtriangleup\{b,c\}} = G_{M,B} \wedge \{b,c\}$.
\end{lemma}


\subsection{Isotropic systems}

\begin{definition}
Let $F=\{0,x,y,z\}$ be a two-dimensional linear space over $\mathbb{F}_2$ defined by the addition
\begin{table}[h]
\centering
\begin{tabular}[b]{|c|cccc|}
\hline
$+$&$0$&$x$&$y$&$z$\\
\hline
$0$&$0$&$x$&$y$&$z$\\
$x$&$x$&$0$&$z$&$y$\\
$y$&$y$&$z$&$0$&$x$\\
$z$&$z$&$y$&$x$&$0$\\
\hline
\end{tabular}.
\end{table}

A bilinear form $\langle\cdot,\,\cdot\rangle\colon F\times F\to\mathbb{F}_2$ is defined by
\begin{table}[h]
\centering
\begin{tabular}[b]{|c|cccc|}
\hline
$\langle\cdot,\,\cdot\rangle$&$0$&$x$&$y$&$z$\\
\hline
$0$&$0$&$0$&$0$&$0$\\
$x$&$0$&$0$&$1$&$1$\\
$y$&$0$&$1$&$0$&$1$\\
$z$&$0$&$1$&$1$&$0$\\
\hline
\end{tabular}.
\end{table}

For a finite set $V$, $F^V$ denotes a set of maps from $V$ to $F$.
For $A,\, B\in F^V$, an addition is defined as
\begin{align*}
(A+B)(v)&\coloneq A(v) + B(v)\qquad\forall v\in V.
\end{align*}
Then, $F^V$ is a linear space on $\mathbb{F}_2$ with this addition.
A bilinear form $\langle\cdot,\,\cdot\rangle\colon F^V\times F^V\to\mathbb{F}_2$ is defined as
\begin{align*}
\langle A,\,B\rangle &\coloneqq \sum_{v\in V} \langle A(v),\,B(v)\rangle.
\end{align*}
In other words, $F^V = \bigoplus_{v\in V}F$.
\end{definition}

\begin{definition}[Isotropic system]
For a finite set $V$, a subspace $W$ of $F^V$ is an \textit{isotropic system} over $V$ if it satisfies
\begin{enumerate}
\item $\forall A,\,B\in W$, $\langle A,\,B\rangle = 0$.
\item $\dim(W) = |V|$.
\end{enumerate}
\end{definition}

\begin{definition}[Girth]
For $A\in F^V$, a \textit{weight} of $A$ is denoted by $|A|\coloneqq|\{v\in V\mid A(v)\ne 0\}|$.
For an isotropic system $W$ over $V$, a \textit{girth} of $W$ is defined as the minimum weight of non-zero vectors in $W$.
\end{definition}

\begin{definition}[Local equivalence]
For isotropic systems $W,\,U$ over $V$, $U$ is \textit{locally equivalent} to $W$ if there exists an invertible linear map $T_v$ over $F$ for each $v\in V$ such that $T\coloneqq \bigoplus_{v\in V}T_v$ is an isomorphism from $U$ to $W$.
%\begin{align*}
%W &= \{ A\in  F^V\mid \exists B\in U,\,A(v) = T_v(B(v))\quad\forall v\in V\}.
%\end{align*}
We denote by $U\sim W$ the fact that $U$ and $W$ are locally equivalent.
\end{definition}
Note that a set of invertible linear maps over $F$ can be regarded as a set of permutations over $\{x,y,z\}$.

\begin{definition}[Isotropic system associated with graph]
For a graph $G$ on a vertex set $V$, define an isotropic system $W_G$ over $V$ that is spanned by
\begin{align*}
A_v(u) &=
\begin{cases}
 x& \text{if } u = v\\
 z& \text{if } u\in N_G(v)\\
 0& \text{otherwise}
\end{cases},\qquad \forall u\in V
\end{align*}
for $v\in V$.
\end{definition}

\begin{definition}[Fundamental graph]
Let $W$ be an isotropic system over $V$.
A graph $G$ is a \textit{fundamental graph} of $W$ if $W$ is locally equivalent to $W_G$.
\end{definition}

Bouchet proved that any isotropic system has a fundamental graph.

\begin{lemma}
Let $W$ be an isotropic system over $V$.
Graphs $G,\,H$ are fundamental graphs of $W$ if and only if $G$ and $H$ are locally equivalent.
\end{lemma}

For $V'\subseteq V$ and $A\in F^V$, a restriction $A|_{V'}\in F^{V'}$ is defined as $A|_{V'}(v)=A(v)$ for $v\in V'$.
\begin{definition}[Minors of isotropic systems]
For an isotropic system $W$ over $V$, $x$-minor, $y$-minor and $z$-minor are defined as
\begin{align*}
W\setminus_x v&\coloneqq\left\{A|_{V\setminus v}\mid A(v)\in\{0,x\}\right\}\\
W\setminus_y v&\coloneqq\left\{A|_{V\setminus v}\mid A(v)\in\{0,y\}\right\}\\
W\setminus_z v&\coloneqq\left\{A|_{V\setminus v}\mid A(v)\in\{0,z\}\right\}.
\end{align*}
The minors of isotropic systems are isotropic systems as well.
\end{definition}

\begin{definition}
For a graph $G=(V,\,E)$,
\begin{align*}
G\setminus_x v &\coloneqq (G\wedge \{v,w\})-v\qquad \text{for some $w\in N_G(v)$}\\
G\setminus_y v &\coloneqq (G*v) - v\\
G\setminus_z v &\coloneqq G - v
\end{align*}
\end{definition}

\begin{lemma}
Let $W$ be an isotropic system over $V$ with a fundamental graph $G$.
%For any $v\in V$, a graph $G\setminus_z v$ is a fundamental graph of $W\setminus_z v$.
Then,
\begin{align*}
W_G \setminus_x v &\sim W_{G\setminus_x v}\\
W_G \setminus_y v &\sim W_{G\setminus_y v}\\
W_G \setminus_z v &= W_{G\setminus_z v}
\end{align*}
\end{lemma}

\section{Sum-decomposable systems}

\begin{definition}[Sum-decomposable system]
Let $M=(V,\,\mathcal{I})$ be a binary matroid associated with a representation matrix $B\in\mathbb{F}_2^{k\times n}$.
Let $C\in\mathbb{F}_2^{(n-k)\times n}$ be a representatiom matrix for the dual matroid $M^*$.
Then, a subspace $W$ of $F^V$ spanned by
\begin{align*}
M_a(v)&\coloneqq \begin{cases}
0&\text{if } B_{a, v} = 0\\
x&\text{if } B_{a, v} = 1
\end{cases}&\forall a\in\{1,\dotsc,k\},\,v\in V\\
M^*_b(v)&\coloneqq \begin{cases}
0&\text{if } C_{b, v} = 0\\
z&\text{if } C_{b, v} = 1
\end{cases}&\forall b\in\{1,\dotsc,n-k\},\,v\in V
\end{align*}
is an isotropic system, and denoted by $xM\oplus zM^*$.
An isotropic system locally equivalent to $xM\oplus zM^*$ is called a \textit{sum-decomposable system}.
\end{definition}

A fundamental graph of a binary matroid $M$ is a fundamental graph of an isotropic system $xM\oplus zM^*$.

\begin{lemma}
The girth of $xM\oplus zM^*$ is equal to $\min\{\girth(M),\,\girth(M^*)\}$.
\end{lemma}
\begin{proof}
%For any $A\in xM\oplus zM^*$, there exists $S,\,T\subseteq V$ such that $A=x1_S+z1_T$.
Let $S\subseteq E$ be the minimum circuit of $M$.
Then, $z1_S$ is in $xM\oplus zM^*$.
Let $S\subseteq E$ be the minimum circuit of $M^*$.
Then, $x1_S$ is in $xM\oplus zM^*$.
This imlpies $\girth(xM\oplus zM^*)\le \min\{\girth(M),\,\girth(M^*)\}$.
Since there is no cancellation in the sum of $x$-vector and $z$-vector, the minimum non-zero weight vector in $xM+zM^*$ is $x$-vector or $z$-vector.
Hence, $\girth(xM\oplus zM^*)= \min\{\girth(M),\,\girth(M^*)\}$.
\end{proof}

For a fundamental graph $G_{M,B}$ of a binary matroid $M$ and its base $B$, the \textit{minimum left-degree (or right-degree)} of $G_{M,B}$ is a minimum degree of vertices in $B$ (or $V\setminus B$), respectively.

\begin{lemma}
Let $M$ be a binary matroid.
%Let $G$ be a fundamental graph of $xM\oplus zM^*$.
Let $B$ be an arbitrary base of $M$.
%Let $G_{M,\,B}$ be a fundamental graph of $M$ for a base $B$ of $M$.
Then, the minimum right-degree of $G_{M,B}$ among all choices of base $B$ of $M$ is $\girth(M)-1$.
Similarly, the minimum left-degree of $G_{M,B}$ among all choices of base $B$ of $M$ is $\girth(M^*)-1$.
\end{lemma}

\begin{corollary}
Let $M$ be a binary matroid.
Let $G$ be a fundamental graph of $xM\oplus zM^*$.
The minimum degree of bipartite graphs that are pivot-equivalent to $G$ is $\min\{\girth(M),\,\girth(M^*)\}-1$.
\end{corollary}


\begin{lemma}
Let $M$ be a binary matroid on $V$.
Then, for $v\in V$.
\begin{align*}
(xM\oplus zM^*)\setminus_x v &= x(M\setminus v) \oplus z (M^* \mathbin{/} v)\\
(xM\oplus zM^*)\setminus_z v &= x(M\mathbin{/} v) \oplus z (M^* \setminus v).
\end{align*}
\end{lemma}

\section{Stabilizer states}

\section*{Acknowledgments}
The work of RM was supported by JST FOREST Program Grant Number JPMJFR216V and JSPS KAKENHI Grant Numbers JP20H04138, JP20H05966 and JP22H00522.


\bibliographystyle{alpha}
\bibliography{biblio}
%\printbibliography

\end{document}
